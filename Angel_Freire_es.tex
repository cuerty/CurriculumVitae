\documentclass[a4paper,11pt]{moderncv}

% moderncv themes
% optional argument are 'blue' (default), 'orange', 'red', 'green', 'grey' and
% 'roman' (for roman fonts, instead of sans serif fonts)
\moderncvtheme[orange]{classic}

% for web links
\definecolor{href}{rgb}{0.2,0.4,0.65}
\newcommand\weblink[2] {{\color{href} \href{#1}{#2}}}
\newcommand\cvskill[2] {\cvline{#1}{
\begin{minipage}[t]{%
  \linewidth}\small #2
\end{minipage}
}}

% character encoding
\usepackage[utf8]{inputenc}

% adjust the page margins
\usepackage[scale=0.85]{geometry}
\AtBeginDocument{\recomputelengths}

% personal data
\firstname{Angel}
\familyname{Freire}
\title{Arquitecto de Software}
\address{Av. Alberdi 4163 8C}{C1407GZI  Capital Federal, Argentina}
\phone{+54 11 6379 9160}
\mobile{+54 911 3005 1170}
\email{cuerty@gmail.com}
\homepage{http://ar.linkedin.com/in/angelfreire}

\nopagenumbers{}

\begin{document}
\maketitle

\section{Objetivos}
Liderar equipos en el desarrollo y entrega de productos de alta calidad. Ser
entrenado en nuevas tecnolog\'ias y metodolog\'ias de trabajo. Conocer
profesionales de distintas industrias y aprender de ellos.

\section{Descripci\'on}
Me gustan los desaf\'ios t\'ecnicos. Creo que el verdadero desaf\'io de
desarrollar software est\'a en definir bien las especificaciones y manejar las
expectativas de los clientes. Me gusta programar bien, utilizando la
herramienta correcta para el problema y no reinventar la rueda. En mi
opini\'on, el formar un buen grupo de trabajo es fundamental para dar las
mejores soluciones posibles. Me agrada usar Open Source, leer el c\'odigo del
software que utilizo, modificarlo y mejorarlo. Estoy especialmente interesado
por los desaf\'ios que involucran grandes vol\'umenes de datos y concurrencia.

\section{Conocimientos t\'ecnicos}
\begin{itemize}
\item Python: Twisted, Tornado, biblioteca est\'andar, Django, Flask, SQL Alchemy. Desarrollo de bibliotecas y m\'odulos utilizando la API de CPython, asi como uso de la misma para embeber Python en otros proyectos. Conocimiento avanzado.
\item PHP: Manejo de bibliotecas PEAR, POO, PHP Unit y frameworks (Artisan, Symphony, CakePHP).  Conocimiento avanzado.
\item C++: Desarrollo de aplicaciones y servicios. Bibliotecas boost, PoCo y QT. Conocimiento avanzado.
\item .NET (C\#, Visual Basic): Windows Forms, Web Forms (Asp.NET), Asp.NET MVC, Unit Testing (NUnit), Mono. Conocimiento b\'asico.
\item Bases de datos: PostgreSQL y MySQL: Conocimiento avanzado. SQL Server: Conocimiento intermedio.
\item Bases de datos no relacionales: Redis, Tokyo Tyrant/Cabinet, MongoDB y CouchDB. Conocimiento intermedio.
\item Linux: Tanto para servidores como para desktop. Conocimiento avanzado.
\item Herramientas de integraci\'on continua y control de versi\'on: Buildbot, Hudson (Jenkins). GIT, Mercurial y Subversion.
\item Message Oriented Brokers: RabbitMQ, Qpid, HornetQ. Conocimiento de los protocolos AMQP y Stomp. Conocimiento avanzado.
\item Otros: GNU Toolchain. Conocimientos de C, Perl, Ruby, sed, awk y Visual Basic. Administraci\'on de Linux, FreeBSD, Solaris y Windows.
\end{itemize}

\pagebreak

\section{Antecedentes Laborales en Relaci\'on de Dependencia}
\cventry{Enero 2011\\Junio 2011}{Lider t\'ecnico de plataformas}
{\weblink{http://www.movile.com}{Movile}}{Buenos Aires, Argentina}{}{%
\begin{itemize}
  \item Document\'e las distintas conexiones existentes.
  \item Dise\~n\'e las estrategias para la integraci\'on de las plataformas de la empresa.
  \item Lider\'e la integraci\'on de las plataformas de mensajer\'ia, suscriptores y cobros en la fusi\'on Movile - Cyclelogic.
\end{itemize}
}
%
\cventry{Febrero 2007\\Diciembre 2010}{Jefe de Arquitectura}
{\weblink{http://www.cyclelogic.com}{Cyclelogic}}{Buenos Aires, Argentina}{}{%
\begin{itemize}
  \item Document\'e las distintas conexiones existentes.
  \item Lider\'e equipos de hasta 12 desarrolladores manejando varios proyectos simult\'aneamente.
  \item Defin\'i los procesos de entrevistas de programadores. Tarea que estuvo a mi cargo durante todo el per\'iodo.
  \item Organic\'e la documentacion t\'ecnica de la empresa utilizando un software de Wiki.
  \item Organic\'e la adopci\'on de herramientas de integraci\'on continua.
  \item Organic\'e la adopci\'on de herramientas de tickets.
  \item Instalaci\'on y administraci\'on de servidores Linux, MySQL y PostgreSQL.
  \item Lider\'e la restructuraci\'on de varias de las plataformas base, asi como su actualizaci\'on y puesta en producci\'on.
  \item Instaur\'e el uso de herramientas de control de versi\'on.
  \item Instaur\'e el realizar code reviews.
  \item Organic\'e la capacitaci\'on de los equipos en distintas tecnolog\'ias y herramientas.
  \item Dise\~n\'e, desarroll\'e e implement\'e distintas plataformas utilizando Python, C++ y PostgreSQL.
  \item Estuve a cargo de la integraci\'on con varios clientes y proveedores con los cuales tuve trato directo.
\end{itemize}
}
%
\cventry{Noviembre 2006\\Enero 2007}{Desarrollador Senior}
{\weblink{http://www.cyclelogic.com}{Cyclelogic}}{Buenos Aires, Argentina}{}{%
\begin{itemize}
  \item Dise\~no y desarrollo de plataforma de mensajer\'ia SMS y MMS en Python y Delphi.
  \item Mantenimiento de plataformas base de la empresa hechas en Delphi y Python.
  \item Instal\'e y configur\'e varios de los servidores Linux y MySQL utilizados.
  \item Escrib\'i la documentaci\'on t\'ecnica de los desarrollos realizados.
\end{itemize}
}
%
\cventry{Enero 2004\\Enero 2005}{Desarrollador Semi Senior}
{\weblink{http://www.iadb.org}{Banco Interamericano de Desarrollo}}{Buenos Aires, Argentina}{}{%
\begin{itemize}
  \item Relevamiento de los requisitos funcionales para sistema de control de becas (viandas alimenticias).
  \item Instal\'e y configur\'e varios servidores Linux tanto para desarrollo y homologaci\'on como para producci\'on.
  \item Desarroll\'e el sistema de control de becas escolares utilizando .NET Framework 1.0.
  \item Realic\'e la documentaci\'on de todos los m\'odulos que compon\'ian a la soluci\'on.
  \item Lider\'e el equipo de deployment e implementaci\'on in-situ.
\end{itemize}
}
%
\cventry{Enero 2004\\Enero 2005}{Desarrollador Senior}
{\weblink{http://www.buenosaires.gov.ar/areas/educacion}{Secretaria de Educaci\'on}}{Buenos Aires, Argentina}{}{%
\begin{itemize}
  \item Configur\'e y administr\'e servidores de bases de datos PostgreSQL, MySQL y OpenLDAP.
  \item Instal\'e y configur\'e varios servidores Linux tanto para desarrollo y homologaci\'on como para producci\'on.
  \item Desarroll\'e el sistema de control de gesti\'on integral utilizando Visual Basic 6.
  \item Desarroll\'e el sistema "Mesa de Entrada" utilizando Visual Basic 6.
\end{itemize}
}

\pagebreak

\section{Antecedentes Laborales como Free Lancer}
\cventry{2000}{Lider t\'ecnico y desarrollador}
{Portal de Internet}{Buenos Aires, Argentina}{}{%
\begin{itemize}
  \item Dise\~n\'e y desarroll\'e un CMS utilizando PHP y MySQL con un backend en Visual Basic para cargar datos.
\end{itemize}
}
%
\cventry{2003}{Lider t\'ecnico y desarrollador}
{Comercio Online}{Internacional}{}{%
\begin{itemize}
  \item Dise\~n\'e y desarroll\'e en PHP y MySQL un CMS integrado con gateways de pago internacionales.
\end{itemize}
}
%
\cventry{2005}{Administrador de sistemas}
{Webmail}{Internacional}{}{%
\begin{itemize}
  \item Instal\'e y configur\'e una soluci\'on de webmail utilizando tecnolog\'ias de FreeBSD y MySQL para HA.
\end{itemize}
}
%
\cventry{2005\\2006}{Lider t\'ecnico y desarrollador}
{Aplicaciones de Optimizacion Web para buscadores (SEO)}{Internacional}{}{%
\begin{itemize}
  \item Dise\~n\'e e implement\'e distintas aplicaciones orientadas a la optimaci\'on web.
  \item Desarroll\'e clientes distribuibles utilizando .NET 2003 y 2005.  (.NET 1.0 y 1.1)
  \item Desarroll\'e servicios web re utilizando bibliotecas programadas en distintos lenguajes. (Ruby, Perl, Python y PHP)
\end{itemize}
}

\section{Educaci\'on Secundaria y Universitaria}
\cventry{2001 -- Inconclusa}{Licenciatura en Ciencias de la Comunicaci\'on Social}{Facultad de Ciencias Sociales (UBA)}{Argentina}{}{}
\cventry{1996 -- 2000}{Bachiller en Comunicaci\'on social}{Escuela Julio Cort\'azar}{Argentina}{}{}

\section{Cursos realizados}
\begin{itemize}
\item Introducci\'on a la Inform\'atica IBM: Generalidades del manejo de sistemas operativos y software de oficina. (En centro IBM certificado)
\item Programaci\'on: Estudio del fundamento y uso pr\'actico de los lenguajes l\'ogicos de programaci\'on. (Profesor Particular)
\end{itemize}

\section{Idiomas}
\cvlanguage{Espa\~nol}{Nativo}{}
\cvlanguage{Ingl\'es}{Nivel intermedio}{Nivel escrito avanzado, oral intermedio.}

\end{document}
